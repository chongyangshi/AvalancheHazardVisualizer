\documentclass[11pt, oneside]{article}   	
\usepackage[left=26mm,top=26mm,right=26mm,bottom=26mm]{geometry}    
\geometry{a4paper}                   		
\usepackage{graphicx}					
\usepackage{amssymb}
\usepackage{cite}
\usepackage{url}
\usepackage[parfill]{parskip}
\setlength{\headsep}{5pt}

\title{\vspace{-1.6cm}Summary of Project Plan (Draft)}
\author{\textit{An app for visualisation of avalanche hazard}}
\date{}							
\begin{document}
\maketitle

\section{Produce Hazard Overlay for Cesium Terrain Map}
Cesium includes a prepared terrain model provided by STK World Terrains \cite{stk-world-terrain}, whose European terrain data was sourced from Digital Elevation Model over Europe (EU-DEM) \cite{eu-dem}. The resolution of EU-DEM is 25 to 30 meters per tile, which is less precise than the 5-meter UK-only terrain model provided by Ordnance Survey OS Terrain 5 \cite{os-5}. However, STK World Terrain allows free public usage \cite{stk-world-terrain}, compared to Edina's educational use-only policy.

It is necessary to devise an algorithm to calculate the avalanche risk of each tile. The risk will be calculated from the following sources:
\begin{itemize}
  \item Avalanche risk forecast dependent on the elevation of each tile in a region, provided by Scottish Avalanche Information Service \cite{sais}, automatically parsed by the \textit{SAISCrawler} as part of the project.
  \item Weighted average tile normal, provided by terrain model.
  \item Average elevation of the tile, provided by terrain model.
\end{itemize}

Existing literatures on avalanche risk assessment such as Rudolf-Miklau \textit{et al.} \cite{91097820150101} and Cappabianca \textit{et al.} \cite{Cappabianca2008193} will be consulted for the design of risk assessment algorithm. It is also worth considering the risk of neighbouring tiles (especially tiles above).

With a risk factor calculated per tile, appropriate colour coding \cite{macdonald1999using} will be applied to the transparent overlay image for each tile. This collection of overlay images will be fed to Cesium and laid on top of tiles in the 3D scene.

\section{Pathfinding and Route Optimisation}

Where schedule permits, this stage could add additional features such as hike route planning to the application; if schedule does not permit, this stage may be skipped.

Essentially, route planning in this 3D model involves a A* search \cite{4082128} optimised for a 3D mesh, with existing research primarily done for AI in video games \cite{cui2011based}, but the underlying principles for this project's model are very similar. However, it is also necessary to take into account the risk factor for each tile (and their neighbours) calculated in the previous stage as a weight.

Route Optimisation could also be expanded to include a user-adjustable risk/distance ratio, as shorter distance may mean a higher risk of avalanche. The user would be able to adjust how much risk they are willing to undertake while shortening their route.

\section{Evaluations on User Experience}

With the aid of an HCI textbook, such as \textit{Interaction Design} \cite{interaction-design}, the application will be evaluated for its usability. If possible, HCI experiments will be conducted with volunteers to obtain evaluation data and feedback.

Limited 3D performance on mobile devices may also be detrimental to user experience, therefore additional optimisations may be required for the application.

\bibliographystyle{IEEEtran}
\small{\bibliography{project}}
\end{document}  