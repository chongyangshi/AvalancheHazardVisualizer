\documentclass{UoYCSproject}

\protect\BEng
\protect\supervisor{Dr William Smith}
\protect\wordcount{0}
\protect\includes{the title page, abstract and body of the report}
\protect\excludes{bibliographies and appendices}
\protect\abstract{This project produces a functional mobile application to visualise potential avalanche hazards along popular winter mountaineering routes within a 3D terrain reality, with a primary focus on Scottish mountains while providing excellent adaptability for other regions with appropriate data sources. Avalanche hazards were projected from a combination of professional avalanche forecasts and spatial analysis of mountain terrain models. Both the accuracy and the usability of the application were evaluated through various methods. The practical and ethical concerns on the application being put into real world use were also considered.}

\usepackage[left=26mm,top=26mm,right=26mm,bottom=26mm]{geometry}    
\geometry{a4paper}                   		
\usepackage{graphicx}					
\usepackage{amssymb}
\usepackage{url}
\usepackage[parfill]{parskip}
\setlength{\headsep}{5pt}

\title{An app for visualisation of avalanche hazard}
\author{Chongyang Shi}
\date{}							
\begin{document}

\maketitle

\chapter{Introduction}

The purpose of this project is to produce a functional mobile application to visualise potential avalanche hazards along popular winter mountaineering routes within a 3D terrain reality, with a primary focus on Scottish mountains while providing excellent adaptability for other regions with appropriate data sources. 

A model for projecting localised avalanche hazards based on carefully sourced and processed avalanche forecast and terrain model data was produced by the project. The accuracy of the model has been evaluated against past avalanche records. The effectiveness of the application guiding the user away from hazardous locations, as well as the application's usability have been evaluated by experiments and test uses conducted with experienced mountaineers.

The project incorporated tools and techniques from various aspects of Computer Science, including software engineering, geographic information system (GIS) modelling, human-computer interaction, computer graphics and algorithms. Many of these tools and techniques are associated with current research in these areas.

The rest of this chapter sets out the objectives and stages of this project, as well as the concerns on ethics associated with the conduction and product of this project.

\section{Objectives of the project}

\section{Stages of the project}

\section{Concerns and Statements on Ethics}

\small{\bibliography{project}}
\end{document}  